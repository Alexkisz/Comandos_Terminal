\documentclass{article}
\usepackage{amsmath}
\usepackage{amsthm}

\title{Comandos Terminal}
\author{Alexis Palomares Olegario}
\date{04 de Marzo de 2025}

\begin{document}

\maketitle


\section{Navegación de directorios}
\begin{itemize}
    \item \texttt{pwd}: muestra la ruta completa del directorio actual.
    \item \texttt{cd}: cambia de directorio (change directory).
    \item \texttt{cd ..}: retrocede un directorio.
    \item \texttt{cd <Nombre carpeta>}: avanza a la carpeta especificada.
    \item \texttt{cd <Nombre carpeta/Nombre carpeta>}: avanza dos carpetas.
    \item \texttt{cd <../..Nombre carpeta>}: retrocede dos veces y luego avanza a la carpeta.
    \item \texttt{start . / explorer .}: abre la carpeta actual (Windows).
\end{itemize}

\section{Listar archivos y directorios}
\begin{itemize}
    \item \texttt{ls}: lista los archivos y directorios en el directorio actual.
    \item \texttt{ls -l}: muestra datos detallados de los archivos.
    \item \texttt{ls -a}: muestra archivos ocultos y no ocultos.
    \item \texttt{ls -la}: muestra toda la información, incluyendo archivos ocultos.
    \item \texttt{ls -l | less}: permite ver los resultados con paginación.
    \item \texttt{ls > salida.txt}: guarda el contenido de \texttt{ls} en un archivo.
    \item \texttt{ls -lh}: muestra el listado con tamaños legibles.
\end{itemize}

\section{Crear y eliminar archivos y carpetas}
\begin{itemize}
    \item \texttt{mkdir <Nombre carpeta>}: crea una nueva carpeta.
    \item \texttt{rmdir <Nombre carpeta>}: elimina una carpeta vacía.
    \item \texttt{rm -r <Nombre carpeta>}: elimina una carpeta y su contenido.
    \item \texttt{rm -rf <Nombre carpeta>}: elimina una carpeta sin confirmación.
    \item \texttt{touch <Nombre archivo>}: crea un archivo vacío.
    \item \texttt{rm <Nombre archivo>}: elimina un archivo.
\end{itemize}

\section{Mover y copiar archivos y carpetas}
\begin{itemize}
    \item \texttt{mv <Origen> <Destino>}: mueve un archivo o carpeta.
    \item \texttt{cp <Origen> <Destino>}: copia un archivo o carpeta.
\end{itemize}

\section{Ver y gestionar contenido de archivos}
\begin{itemize}
    \item \texttt{cat <Nombre archivo.txt>}: imprime el contenido del archivo.
    \item \texttt{cat <Nombre archivo.txt> | grep "error"}: busca la palabra "error" dentro del archivo.
    \item \texttt{cat desordenado\_duplicados.txt | sort | uniq}: ordena y elimina duplicados.
    \item \texttt{sort < desordenado.txt}: imprime el contenido ordenado.
\end{itemize}


\section{Administración y limpieza}
\begin{itemize}
    \item \texttt{clear}: limpia la consola.
    \item \texttt{history}: muestra el historial de comandos.
    \item \texttt{sudo apt-get clean}: elimina archivos temporales.
    \item \texttt{sudo apt-get autoremove}: elimina paquetes innecesarios.
\end{itemize}

\section{Variables de entorno y PATH}
\begin{itemize}
    \item \texttt{export PALABRA=ls}: asigna \texttt{"ls"} a la variable \texttt{PALABRA} y la hace accesible para otros procesos.
    \item \texttt{echo \$PALABRA}: muestra el contenido de la variable.
    \item \texttt{export PATH="\$PATH:/home/usuario/bin"}: agrega \texttt{/home/usuario/bin} al \texttt{PATH}, permitiendo ejecutar scripts desde ahí.
    \item \texttt{echo \$PATH}: muestra las rutas actuales en \texttt{PATH}.
\end{itemize}

\section{Búsqueda y filtrado}
\begin{itemize}
    \item \texttt{find <ruta> -name "<nombre>"}: busca archivos por nombre.
    \item \texttt{grep "<texto>" <archivo>}: busca texto dentro de un archivo.
    \item \texttt{grep -r "<texto>" <carpeta>}: busca texto en todos los archivos de una carpeta.
    \item \texttt{which <comando>}: muestra la ruta del ejecutable.
\end{itemize}

\section{Permisos y Archivos en Linux}

\subsection{Permisos y usuarios}
\begin{itemize}
    \item \texttt{chmod <permisos> <archivo>}: cambia los permisos.
    \item Ejemplo: \texttt{chmod 755 archivo} (\texttt{rwxr-xr-x}).
    \item \texttt{chown <usuario>:<grupo> <archivo>}: cambia el propietario y grupo.
    \item \texttt{sudo su}: cambia a usuario root.
\end{itemize}

\subsection{Tipos de archivos (primer carácter en \texttt{ls -l})}
\begin{itemize}
    \item \texttt{-}: archivo normal.
    \item \texttt{d}: directorio.
    \item \texttt{l}: enlace simbólico.
\end{itemize}

\subsection{Permisos y sus valores numéricos}
\begin{center}
\begin{tabular}{|c|c|c|}
\hline
\textbf{Permiso} & \textbf{Número} & \textbf{Significado} \\
\hline
r (lectura) & 4 & Leer el contenido. \\
w (escritura) & 2 & Modificar el contenido. \\
x (ejecución) & 1 & Ejecutar el archivo. \\
- (sin permiso) & 0 & Ningún permiso. \\
\hline
\end{tabular}
\end{center}



\section{Información del sistema}
\begin{itemize}
    \item \texttt{uname -a}: muestra información completa del sistema.
    \item \texttt{df -h}: muestra el espacio en disco libre y usado.
    \item \texttt{du -sh <carpeta>}: muestra el tamaño de una carpeta.
    \item \texttt{top / htop}: muestra los procesos activos y uso de recursos.
    \item \texttt{free -h}: muestra el uso de memoria RAM.
\end{itemize}

\section{Redes y conectividad}
\begin{itemize}
    \item \texttt{ping <dominio>}: verifica la conectividad.
    \item \texttt{ifconfig / ip a}: muestra las interfaces de red.
    \item \texttt{netstat -tuln}: muestra los puertos abiertos.
    \item \texttt{curl <URL>}: realiza solicitudes HTTP.
\end{itemize}

\section{Compresión y descompresión}
\begin{itemize}
    \item \texttt{tar -cvf <archivo.tar> <carpeta>}: crea un .tar.
    \item \texttt{tar -xvf <archivo.tar>}: extrae un .tar.
    \item \texttt{tar -czvf <archivo.tar.gz> <carpeta>}: crea un .tar.gz.
    \item \texttt{tar -xzvf <archivo.tar.gz>}: descomprime un .tar.gz.
    \item \texttt{zip <archivo.zip> <archivo\_o\_carpeta>}: crea un .zip.
    \item \texttt{unzip <archivo.zip>}: descomprime un .zip.
\end{itemize}



\section{Git y ayuda de comandos}
\begin{itemize}
    \item \texttt{git init}: inicia un proyecto Git.
    \item \texttt{<comando> --help}: muestra el manual breve del comando.
\end{itemize}

\section{Combinaciones útiles}
\begin{itemize}
    \item \texttt{du -h --max-depth=1}: muestra el tamaño de carpetas.
    \item \texttt{ps aux | grep <proceso>}: busca procesos activos.
    \item \texttt{!<número>}: ejecuta un comando del historial por número.
\end{itemize}

\section{Alias para agilizar comandos}
\begin{itemize}
    \item \texttt{alias ll='ls -la'}: crea un alias para listar todo con detalles.
    \item \texttt{alias cls='clear'}: crea un alias para limpiar la consola.
\end{itemize}

\end{document}
